\documentclass{llncs}
\usepackage{times}
\usepackage[T1]{fontenc}
\usepackage[brazilian]{babel}
\usepackage[utf8]{inputenc}
\usepackage[T1]{fontenc}
\usepackage{indentfirst}
%\usepackage[applemac]{inputenc}
\usepackage{a4}
%\usepackage[margin=3cm,nohead]{geometry}
\usepackage{epstopdf}
\usepackage{graphicx}
\usepackage{fancyvrb}
\usepackage{amsmath}
%\renewcommand{\baselinestretch}{1.5}

\begin{document}
\mainmatter
\title{Rede Móvel: 4G para 5G}

\titlerunning{Rede Móvel: 4G para 5G}

\author{Paulo Silva Sousa \and Carlos Miguel Luzia de Carvalho \and Ruben César Ferreira Lucas}

\authorrunning{Paulo Silva Sousa \and Carlos Miguel Luzia de Carvalho \and Ruben César Ferreira Lucas}

\institute{
Universidade do Minho, Departamento de Informática, 4710-057 Braga, Portugal\\
e-mail: \{a89465,a89605,a89487\}@alunos.uminho.pt
}

\date{}
\bibliographystyle{splncs}

\maketitle
\begin{abstract}
Neste ensaio escrito vamos abordar as diferenças entre as tecnologias de rede móvel, principalmente entre 4G e 5G. Além disso, iremos tratar das aplicações desta útlima tecnologia em várias áreas cientificas e no quotidiano.
\end{abstract}

\section{Introdução}
Nos últimos anos, com a evolução tecnológica que estamos a presenciar, tem havido um enorme aumento nas diversas formas de comunicação e partilha de informação, bem como a sua procura. Como tal, surgiu um problema: a atual geração de rede móvel (4G) não está a conseguir acompanhar o ritmo de transmissão de dados atual.

Para isso, surge a necessidade de uma nova tecnologia: 5G. Esta tecnologia representa a quinta geração de rede móvel e promete um aumento na largura de banda e na velocidade, bem como uma diminuição da latência.

\section{Evolução da rede móvel}

Cada geração da rede móvel - abreviada por um G - constuma surgir com um intervalo de 10 anos do sucessor e trazer mudanças significativas na capacidade e velocidade de transporte de dados, bem como na latência. Com isso, acreditamos que a quinta geração não será diferente.

\subsection{Evolução inicial}

A primeira geração (1G) da rede móvel surgiu no início da década de 1980 foi baseada no sistema analógico. A grande vantagem desta geração foi a introdução dos telefones sem fio. Por outro lado, apenas era possivel a comunicação por voz e a capacidade e qualidade de transmissão eram bastante fracas.

Com a chegada da segunda geração (2G) no final da década de 1980 tivemos a chegada da tecnologia de sinalização digital de dados em banda baixa à rede móvel. Com isto, tivemos um aumento na velocidade de transmissão de dados de 2.4kbps para 64kbps. Além disso, tornou-se possível a transmissão, não só de voz, mas também de dados, porém com fraca qualidade.

Na terceira geração (3G), que chegou no início da decada de 2000, tivemos um dos maiores saltos em termos de velocidade de transmissão de dados, passando de 64kpbs para 2Mbps. Isto possibilitou pela primeira vez a transmissão de vídeo e imagem e mobilidade na internet a velocidades elevadas.

Por último, temos a quarta geração da rede móvel (4G), que surgiu no início da década de 2010. Esta geração trouxe, mais uma vez, aumentos significativos na velocidade de transmissão e uma grande diminuição na latência. Esta tecnologia deu-nos o mundo como o conhecemos hoje em dia, com acesso praticamente instantaneo à internet, permitindo assim a obtenção de uma enorme quantidade de informação.

\clearpage

\subsection{Do 4G para o 5G}

\subsection{Desvantagens da evolução para o 5G}

\section{Aplicações do 5G}

CUSTOS

\section{Conclusões}
Neste trabalho...

\begin{thebibliography}{1}
\bibitem{}
J. Agrawal, Rakesh Patel, P. Mor, P. Dubey, J. Keller
\newblock {"Evolution of Mobile Communication Network: from 1G to 4G".} (2015)

\bibitem{}
Larissa de Souza Pereira Rosa, Renan Góes Barcelos, Yago Pereira Pradoe Yan Real
\newblock {"Aplicações do 5G em Internet das Coisas".}
\end{thebibliography}

\end{document}