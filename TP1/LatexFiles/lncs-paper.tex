\documentclass{llncs}
\usepackage{times}
\usepackage[T1]{fontenc}
\usepackage[brazilian]{babel}
\usepackage[utf8]{inputenc}
\usepackage[T1]{fontenc}
\usepackage{indentfirst}
%\usepackage[applemac]{inputenc}
\usepackage{a4}
%\usepackage[margin=3cm,nohead]{geometry}
\usepackage{epstopdf}
\usepackage{graphicx}
\usepackage{fancyvrb}
\usepackage{amsmath}
%\renewcommand{\baselinestretch}{1.5}

\begin{document}
\mainmatter
\title{Rede Móvel: 4G para 5G}

\titlerunning{Rede Móvel: 4G para 5G}

\author{Paulo Silva Sousa \and Carlos Miguel Luzia de Carvalho \and Ruben César Ferreira Lucas}

\authorrunning{Paulo Silva Sousa \and Carlos Miguel Luzia de Carvalho \and Ruben César Ferreira Lucas}

\institute{
Universidade do Minho, Departamento de Informática, 4710-057 Braga, Portugal\\
e-mail: \{a89465,a89605,a89487\}@alunos.uminho.pt
}

\date{}
\bibliographystyle{splncs}

\maketitle
\begin{abstract}
Neste ensaio escrito vamos abordar as diferenças entre as tecnologias de rede móvel, principalmente entre 4G e 5G. Além disso, iremos tratar das aplicações desta útlima tecnologia em várias áreas cientificas e no quotidiano.
\end{abstract}

\section{Introdução}
Nos últimos anos, com a evolução tecnológica que estamos a presenciar, tem havido um enorme aumento nas diversas formas de comunicação e partilha de informação, bem como a sua procura. Como tal, surgiu um problema: a atual geração de rede móvel (4G) não está a conseguir acompanhar o ritmo de transmissão de dados atual.

Para isso, surge a necessidade de uma nova tecnologia: 5G. Esta tecnologia representa a quinta geração de rede móvel e promete um aumento na largura de banda e na velocidade, bem como uma diminuição da latência.

\section{Evolução da rede móvel}

Cada geração da rede móvel - abreviada por um G - constuma surgir com um intervalo de 10 anos do sucessor e trazer mudanças significativas na capacidade e velocidade de transporte de dados, bem como na latência. Com isso, acreditamos que a quinta geração não será diferente.

A atual geração de rede móvel utilizada é a quarta. Esta tecnologia apresenta cerca de 200 mbps de bandalarga, 25mbps de velocidade e 20-30 milisegundos de latência. 

%4G
Lançamento: 2006-2010
Bandalarga: 200mbps
Latência: 20-30 milisegundos
Velocidade: 25mbps

%5G 
Lançamento: 2020
Bandalarga: 1gbps
Latência: 10 milisegundos
Velocidade: 200-400mbps

\section{Aplicações desta tecnologia}

\section{Conclusions}
Neste trabalho...

\begin{thebibliography}{1}
\bibitem{Zadeh65}
Zadeh, L.:
\newblock {Fuzzy sets} (1965)

\bibitem{Nguyen99}
Nguyen, H., Walker, E.:
\newblock {First course in fuzzy logic}.
\newblock {Boca Raton: Chapman and Hall/CRC Press} (1999)
\end{thebibliography}

\end{document}